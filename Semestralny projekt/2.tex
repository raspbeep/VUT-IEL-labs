\section{Príklad č.~2}

\subsection{Zadanie}Stanovte napätie $U_{{R}_3}$ a prúd $I_{{R}_3}$. Použite metódu Théveninovej vety.

\begin{table}[ht]
	\centering
	\begin{tabular}{|c|c|c|c|c|c|c|c|}
		\hline
		sk. & $U_{1}$~[V] & $R_{1}~[\Omega]$ & $R_{2}~[\Omega]$ & $R_{3}~[\Omega]$ & $R_{4}~[\Omega]$ & $R_{5}~[\Omega]$ & $R_{6}~[\Omega]$\\
		\hline
		B&100&50&310 &610 &220 &570 & 200\\
		\hline
	\end{tabular}
\end{table}

\subsection{Riešenie}

\begin{figure}[h!]
\begin{circuitikz} \draw
(0,5.25) to[dcvsource, v_=$U_1$](0,0)
(0,5.25) -- (1.5,5.25)
(1.5,7) to[R, *-*, l_=$R_1$](5,7)
(5,7) to[R, *-*, l_=$R_4$](8.5,7)
(1.5,7) -- (1.5,3.5)

(5,7) to[R, *-*, l=$R_3$, i=$I_3$, v=$U_3$](5,3.5)
(1.5,3.5) to[R, *-*, l_=$R_2$](5,3.5)
(8.5,7) to[R, -*, l_=$R_5$, ](8.5,3.5)
(8.5,4.5) -- (8.5,3.5)
(5,3.5) to[R, *-*, l=$R_6$](8.5,3.5)
(0,0) -- (8.5,0)
(8.5,3.5) -- (8.5,0)
;
\end{circuitikz}
\centering
\caption{Pôvodný obvod}
\end{figure}

\begin{figure}[h!]
\begin{circuitikz} \draw
(0,5.25) to[dcvsource, v_=$U_1$](0,0)
(0,5.25) -- (1.5,5.25)
(1.5,7) to[R, *-*, l_=$R_1$](5,7)
(5,7) to[R, *-*, l_=$R_4{_5}$](8.5,7)
(1.5,7) -- (1.5,3.5)

node[label={above:A}] (A) at (5,7){} node[label={below:B}] (B) at (5,3.5){} (A) to[open, v=$U_i$] (B)

(1.5,3.5) to[R, *-*, l_=$R_2$](5,3.5)
(8.5,7) -- (8.5,3.5)

(5,3.5) to[R, *-*, l=$R_6$](8.5,3.5)
(0,0) -- (8.5,0)
(8.5,3.5) -- (8.5,0)
;
\end{circuitikz}
\centering
\caption{Obvod bez $R_3$}
\end{figure}


\begin{equation*}
\begin{aligned}
R_4{_5} &=R_4+R_5 \\
\end{aligned}
\end{equation*}

\noindent
\textit{Napätie $U_i$ je rovné rozdielu napätí pred rezistormi $R_6$ a $R_4{_5}$(proti zemi). Pre výpočet je možné použiť postup pre napäťový delič.}

\begin{equation*}
\begin{aligned}
U_{B} &=\frac{R_6}{R_2+R_6}U_1 \\ \\
U_{A} &=\frac{R_4{_5}}{R_1+R_4{_5}}U_1 \\ \\
U_i &= \abs{U_{R_A}-U_{R_B}} \\
\end{aligned}
\end{equation*}

\begin{figure}[h!]
\begin{circuitikz} \draw

(0,0) -- (0,4.25)
(0,4.25) -- (1.5,4.25)
(1.5,6) to[R, l_=$R_1$](5,6)

node[label={above:A}] (A) at (5,6){} node[label={below:B}] (B) at (5,2.5){} (A) to[open, v=$U_i$] (B)


(1.5,6) -- (1.5,2.5)
(1.5,2.5) to[R, -, l=$R_2$](5,2.5)
(8.5,6) -- (8.5,2.5)
(5,6) to[R, *-, l_=$R_4{_5}$, ](8.5,6)

(5,2.5) to[R, *-, l=$R_6$](8.5,2.5)
(0,0) -- (10,0)

(10,0) -- (10,4.25)
(8.5,4.25) -- (10,4.25)
;
\end{circuitikz}
\centering
\caption{Nahradenie zdroja skratom}
\end{figure}

\textit{Ďalej je potrebné zistiť odpor $R_i$ medzi bodmi A a B. Obr. 10 sa dá ešte zjednodušiť.}

\begin{figure}[h!]
\begin{circuitikz} \draw

(1.5,0) -- (3.5,0)
(3.5,0) -- (3.5,0.5)
(1.5,0) -- (1.5,0.5)
(0.5,0.5) -- (2.5,0.5)
(0.5,0.5) to[R, l_=$R_2$](0.5,2.5)
(2.5,0.5) to[R, l_=$R_6$](2.5,2.5)
(0.5,2.5) -- (2.5,2.5)
(0.5,2.5) to[R, l_=$R_1$](0.5,4.5)
(2.5,2.5) to[R, l_=$R_4{_5}$](2.5,4.5)
(0.5,4.5) -- (2.5,4.5)
(1.5,4.5) -- (1.5,5)
(1.5,5) -- (3.5,5)
(3.5,5) -- (3.5,4.5)

node[label={below:A}] (A) at (3.5,4.5){} node[label={above:B}] (B) at (3.5,0.5){}
(A) to[open, v^>=$R_i$] (B)
;
\end{circuitikz}
\centering
\caption{Zistenie $R_i$}
\end{figure}


\begin{equation*}
\begin{aligned}
R_1{_4}{_5} &=\frac{R_1R_4{_5}}{R_1+R_4{_5}} \\ \\
R_2{_6} &=\frac{R_2R_6}{R_2+R_6} \\ \\ 
\end{aligned}
\end{equation*}


\begin{figure}[h!]
\begin{circuitikz} \draw
(1.5,0) -- (3.5,0)
(3.5,0) -- (3.5,0.5)
(1.5,0) -- (1.5,0.5)
(1.5,0) to[R, l_=$R_2{_6}$](1.5,2.5)
(1.5,2.5) to[R, l_=$R_1{_4}{_5}$](1.5,5)
(1.5,5) -- (3.5,5)
(3.5,5) -- (3.5,4.5)
node[label={below:A}] (A) at (3.5,4.5){}
node[label={above:B}] (B) at (3.5,0.5){}
(A) to[open, v^>=$R_i$] (B)
;
\end{circuitikz}
\centering
\caption{Zistenie $R_i$}
\end{figure}

\begin{equation*}
\begin{aligned}
R_i &= R_1{_2}{_4}{_5}{_6}=R_1{_4}{_5} + R_2{_6} \\ \\ 
\end{aligned}
\end{equation*}

\begin{figure}[h!]
\begin{circuitikz} \draw
(0,3) to[dcvsource, v_=$U_1$](0,0)
(0,3) to[R, l=$R_i$](3,3)
(3,3) to[R, l=$R_3$, i=$I_3$, v_=$U_3$](3,0)
(0,0) -- (3,0)
;
\end{circuitikz}
\centering
\caption{Výpočet $U_{{_R}_3}$}
\end{figure}

\begin{equation*}
\begin{aligned}
I_{_R{_3}} &= \frac{U_i}{R_i+R_3}  \\ \\ 
I_{_R{_3}} &= \frac{\abs{\frac{R_6}{R_2+R_6}U_1-\frac{R_4+R_5}{R_1+R_4+R_5}U_1}}{\frac{R_1(R_4+R_5)}{R_1+R_4+R_5} + \frac{R_2R_6}{R_2+R_6}+R_3}\approx70.424mA  \\ \\ 
U_{_R{_3}} &= I_3R_3\approx42.9586V \\ \\
\end{aligned}
\end{equation*}